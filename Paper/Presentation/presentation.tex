\documentclass[14pt]{beamer}
\usepackage[T2A]{fontenc}
\usepackage[utf8]{inputenc}
\usepackage[english,russian]{babel}
\usepackage{amssymb,amsfonts,amsmath,mathtext}
\usepackage{cite,enumerate,float,indentfirst}
\usepackage{svg}

\usepackage{listings}

\lstdefinelanguage{Synthesizer}
{
  morekeywords={ Int, String, Bool, record, Collection, query, update, or, and, map, filter, mapFilter, foldr },
  morecomment=[l]{--}, % l is for line comment
  morecomment=[s]{/*}{*/}, % s is for start and end delimiter
  morestring=[b]" % defines that strings are enclosed in double quotes
}

\lstdefinelanguage{CollectionLang}
{
  morekeywords={ Int, String, Bool, List, HashMap, HashSet, FenwickTree, Array, Heap, Set, Map },
  morecomment=[l]{--}, % l is for line comment
  morecomment=[s]{/*}{*/}, % s is for start and end delimiter
  morestring=[b]" % defines that strings are enclosed in double quotes
}

\lstdefinelanguage{SyGuSLang}
{
  morekeywords={ Int, String, Bool, List, Map, Set, Obj, Func, map, filter, mapFilter },   morecomment=[s]{/*}{*/}, % s is for start and end delimiter
  morestring=[b]" % defines that strings are enclosed in double quotes

}


\lstdefinelanguage{SolutionLang}
{
  morekeywords={ Int, String, Bool, List, Map, Set, filter, map, mapFilter, where, let},
  morecomment=[l]{--}, % l is for line comment
  morecomment=[s]{/*}{*/}, % s is for start and end delimiter
  morestring=[b]" % defines that strings are enclosed in double quotes
}

\lstdefinestyle{Synth} {
    language        = Synthesizer,
    basicstyle      = \footnotesize\ttfamily,
    identifierstyle = \color{black},
    commentstyle    = \color{red},
    keywordstyle    = \color{blue},
    stringstyle     = \color{red},
    extendedchars   = true,
    tabsize         = 4,
    showspaces      = false,
    showstringspaces = false,
    breakautoindent = true,
    flexiblecolumns = true,
    keepspaces      = true,
    stepnumber      = 0,
    xleftmargin     = 0pt
}

\lstset{
    style=Synth,
    breaklines=false,
    frame=single,
    literate ={~~}{$\sim$}1,
}


\usepackage{parcolumns}


\graphicspath{{images/}}

\usetheme{Pittsburgh}
\usecolortheme{whale}

\setbeamercolor{footline}{fg=blue}
\setbeamertemplate{footline}{
  \leavevmode%
  \hbox{%
  \begin{beamercolorbox}[wd=.333333\paperwidth,ht=2.25ex,dp=1ex,center]{}%
    Артем Мухин
  \end{beamercolorbox}%
  \begin{beamercolorbox}[wd=.333333\paperwidth,ht=2.25ex,dp=1ex,center]{}%
    Санкт-Петербург, 2017
  \end{beamercolorbox}%
  \begin{beamercolorbox}[wd=.333333\paperwidth,ht=2.25ex,dp=1ex,right]{}%
  Стр. \insertframenumber{} из \inserttotalframenumber \hspace*{2ex}
  \end{beamercolorbox}}%
  \vskip0pt%
}

\newcommand{\itemi}{\item[\checkmark]}

\title{\small{Синтез структур данных}}
\author{\small{%
\emph{Выступающий:}~Артем Мухин, 344 гр.\\%
\emph{Руководитель:}~Дмитрий Мордвинов}\\%
\vspace{30pt}%
СПбГУ
\vspace{20pt}%
}
\date{\small{Санкт-Петербург, 2017}}

\begin{document}

\maketitle

\begin{frame}
\frametitle{Синтез программ}
Синтез программ --- автоматический поиск программы, удовлетворяющей ограничениям, например:
\begin{itemize}
    \item Формальной спецификации
    \item Примерам ввода-вывода
    \item Описанию на естественном языке
    \item Синтаксическому шаблону
\end{itemize}
\end{frame}

\begin{frame}
\frametitle{Мотивация}
\begin{itemize}
    \item Структуры данных используются повсеместно
    \item Зачастую нужны структуры, не содержащиеся явно в стандартных библиотеках
    \begin{itemize}
        \item Композиции нескольких стандартных структур
        \item Стандартные структуры с особыми свойствами
    \end{itemize}
    \item Программист часто ошибается и выбирает не самые оптимальные варианты
    \item Хочется, чтобы программист мог только описать требуемую структуру,
    а компьютер сам синтезировал оптимальную реализацию
\end{itemize}
\end{frame}


\begin{frame}[fragile]
\frametitle{Пример-1}
\begin{itemize}
  \item Нужна структура данных для хранения графа в виде рёбер и быстрого ответа на запрос "Какие рёбра инцидентны данной вершине?".
\end{itemize}
Спецификация:
\begin{lstlisting}[mathescape=true]
record Edge { src: Int, dst: Int }
graph: Collection<Edge>
query findEdges(node: Int) =
    filter ($\lambda$e. e.src = node or e.dst = node) graph
\end{lstlisting}
\end{frame}

\begin{frame}[fragile]
\frametitle{Пример-2}
Возможные решения:
\begin{itemize}
    \item Связный список. Выполнение запроса за O(n).
    \item Два \texttt{HashMap<Int, List<Edge>}:
    \begin{itemize}
        \item пары вида (src, все ребра из src)
        \item пары вида (dst, все ребра в dst)
    \end{itemize}.
    Выполнение запроса за O(1).
\end{itemize}
Синтезатор перебирает все возможные «решения» и выбирает из них оптимальное.
\end{frame}

\begin{frame}[fragile]
\frametitle{Пример-3}
В итоге синтезатор генерирует такой код:
\begin{lstlisting}[language=Java]
public class Graph {
    public static class Edge {
        public Edge(int src, int dst) { ... }
    }
    
    public Graph() { ... }
    public void add(Edge e) { ... }
    public void remove(Edge e) { ... }
    public Iterator<Record> findEdges(int node) { ... }
}
\end{lstlisting}

\end{frame}


\begin{frame}
\frametitle{Существующее решение -- Cozy (PLDI'16)}
Cozy умеет решать только небольшое подмножество довольно простых задач (поиск элементов, удовлетворяющих предикату).\\ \\
Cozy \textbf{не} умеет работать с:
\begin{itemize}
\item Более сложными комбинаторами: map, fold
\item Арифметикой
\item Нестандартными коллекциями
\end{itemize}
\end{frame}


\begin{frame}
\frametitle{Постановка задачи}
\begin{itemize}
    \item Спроектировать язык спецификации
    \item Разработать алгоритм синтеза
    \item Разработать прототип ядра синтезатора
\end{itemize}
\end{frame}


\begin{frame}[fragile]
\frametitle{Структуры данных-1}
\begin{lstlisting}[language=CollectionLang, mathescape=true]
* List<T>, Array<T>, Set<T>, Heap<T>
...
* Map<K, V>
Map.build         :: (t $\to$ (K, V)) $\to$ [t] $\to$ Map<K, V>
Map.get           :: K $\to$ Map<K, V> $\to$ V
Map.add           :: K $\to$ V $\to$ Map<K, V> $\to$ Map<K, V>
Map.size          :: Map<K, V> $\to$ Int

* FenwickTree<T>
FenwickTree.build :: (T$\to$T') $\to$ Array<T> $\to$ FenwickTree<T> 
FenwickTree.eval  :: Int $\to$ Int $\to$ T'
FenwickTree.get   :: Int $\to$ T
\end{lstlisting}

\end{frame}


\begin{frame}
\frametitle{Структуры данных-2}
% Любые преобразования над коллекциями можно выразить через map, filter и foldr.
\small{
{\color{blue}Map}.build f ks $\sim$ {\color{blue}map} ($\lambda$k. (k, f k)) ks\\
{\color{blue}Map}.get key xs $\sim$ {\color{blue}map} snd (filter ($\lambda$(k, v). k = key) xs) \\
... \\ \\
}

\small{
{\color{red}filter} p ({\color{brown}map} f xs)           =  {\color{blue}mapFilter} (p . f) f xs \\
{\color{brown}map} f ({\color{red}filter} p xs)           =  {\color{blue}mapFilter} p     f xs \\
{\color{red}filter} p' ({\color{blue}mapFilter} p f xs)  =  {\color{blue}mapFilter} ($\lambda$x. p x $\And$ p' x) f xs \\
{\color{brown}map} f' ({\color{blue}mapFilter} p f xs)     =  {\color{blue}mapFilter} p (f' . f) xs \\
{\color{blue}mapFilter} p f ({\color{red}filter} p' xs)  =  {\color{blue}mapFilter} ($\lambda$x. p x $\And$ p' x) f xs \\
{\color{blue}mapFilter} p f ({\color{brown}map} f' xs)     =  {\color{blue}mapFilter} (p . f') (f . f') xs \\
}
\\
\begin{itemize}
\item Синтезатор строит решение в терминах операций над коллекциями
\item Для верификации решение сначала переписывается в терминах {\color{blue}mapFilter}-преобразований над списками
\end{itemize}
\end{frame}



\begin{frame}[fragile]
\frametitle{Пример преобразования}
Решение в терминах операций над коллекциями:
\begin{lstlisting}[language=SolutionLang, mathescape=true]
let table: Map<Int, List<Edge>> = Map.build ($\lambda$e.
    (e.src, filter ($\lambda$e'. e'.src = e.src) graph)) graph
findEdges n = Map.get n table
\end{lstlisting}

Редукция к {\color{blue}mapFilter}:
\begin{lstlisting}[language=SolutionLang, mathescape=true]
findEdges n
~~ map snd (filter ($\lambda$(src, _). src = n) (map ($\lambda$e.
    (e.src, filter ($\lambda$e'. e'.src = e.src) graph)) graph))
~~ mapFilter ($\lambda$e. e.src = n) (filter ($\lambda$e'.
    e'.src = e.src) graph) graph
\end{lstlisting}
\end{frame}


\begin{frame}
\frametitle{Перебор и верификация кандидатов}
\begin{itemize}
    \item Явный перебор и проверка с помощью SMT-решателей (подход Cozy и др.)
    \item \textbf{Сведение задачи перебора и верификации к SyGuS}
\end{itemize}
\end{frame}


\begin{frame}
\frametitle{Syntax-Guided Synthesis (SyGuS)}
SyGuS-синтезаторы строят программы, удовлетворяющие:
\begin{itemize}
    \item синтаксической спецификации (грамматике)
    \item семантической спецификации (формуле логики первого порядка)
\end{itemize}
SyGuS Competition проводится каждый год начиная с 2014 г.
\end{frame}


\begin{frame}[fragile]
\frametitle{Построение грамматики для SyGuS}
\begin{lstlisting}[language=SyGuSLang, mathescape=true]
List<Edge>     ::= graph | Map<K, List<Edge>>.get
Map<K,V>.get ::= map snd (filter($\lambda$(k,v).k=Obj<K>) Map<K,V>)
Map<K,V>       ::= Map<K,V>.build
Map<K,V>.build ::= map ($\lambda$k. (k, Func<K,V> k)) List<K>
...
\end{lstlisting}
Специализируем и редуцируем грамматику:
\begin{lstlisting}[language=SyGuSLang, mathescape=true]
map snd (filter ($\lambda$(k,v). k = Obj<K>) Map<K, List<Edge>>) $\sim$

map snd (filter ($\lambda$(k,v). k = Obj<K>) (map ($\lambda$k.
    (k, Func<K,List<Edge>> k)) List<K>)) $\sim$

mapFilter ($\lambda$(k,v). k = Obj<K>) snd (map ($\lambda$k.
    (k, Func<K,List<Edge>>)) List<K) $\sim$

mapFilter ($\lambda$(k,v). k = Obj<K>) Func<K,List<Edge>> List<K>
\end{lstlisting}
\end{frame}


\begin{frame}[fragile]
\frametitle{Построение ограничений для SyGuS}

\begin{lstlisting}[language=SyGuSLang, mathescape=true]
/* specification */
mapFilter ($\lambda$e. e.src = n) id graph
/* candidate */
mapFilter ($\lambda$(k,v).k = Obj<Int>) Func<Int,List<Edge>> graph
\end{lstlisting}


\begin{lstlisting}[language=SyGuSLang, mathescape=true]
mapFilter p f xs $\sim$ mapFilter p' f' xs
$\forall x: p(x) \leftrightarrow p'(x) \And p(x) \rightarrow (f(x) = f'(x))$
\end{lstlisting}

В итоге SyGuS приходит решению:
\begin{lstlisting}[language=SyGuSLang, mathescape=true]
mapFilter ($\lambda$e. e.src = n) ($\lambda$e. filter ($\lambda$e'.
    e'.src = e.src) graph) graph
\end{lstlisting}

Которое эквивалентно:
\begin{lstlisting}[language=SolutionLang, mathescape=true]
Map.get n (Map.build ($\lambda$e. (e.src, filter ($\lambda$e'.
    e'.src = e.src) graph) ) graph)
\end{lstlisting}
\end{frame}



\begin{frame}
\frametitle{Реализация}
Парсер языка спецификации реализован с помощью монадических парсер-комбинаторов (библиотека \textbf{parsec} языка Haskell)
\begin{itemize}
    \item Парсер $=$ примитивные парсеры $+$ комбинаторы парсеров
    \item Преимущества: простота, гибкость, выразительность
\end{itemize}
\end{frame}


\begin{frame}
\frametitle{Реализация}
Для языка спецификации реализованы:
\begin{itemize}
    \item Pretty-printer
    \item Редукция (map, filter $\to$ mapFilter)
    \item Вывод типов (основанный на алгоритме Хиндли-Милнера)
    \item Сведение к SyGuS
    \item Конвертация в JSON для последующей обработки генератором кода
\end{itemize}
\end{frame}


\begin{frame}
\frametitle{Результаты}
\begin{itemize}
    \item Спроектирован декларативный язык спецификации структур данных
    \item Разработан алгоритм синтеза на основе редукции и сведения к SyGuS
    \item Разработан прототип ядра синтезатора, включающий в себя: парсер, вывод типов, редукцию, генерацию программ, конвертацию программ в JSON, сведение к SyGuS
\end{itemize}
\end{frame}

\begin{frame}
\frametitle{Архитектура синтезатора}
\begin{center}
  \makebox[\textwidth]{\includegraphics[width=\paperwidth]{diag.png}}
\end{center}
\end{frame}


\begin{frame}
\frametitle{Планы на будущее}
\begin{itemize}
    \item Добавить поддержку fold
    \item Добавить больше структур данных и операций над ними
    \item Сравнить полученный синтезатор с существующими
\end{itemize}
\end{frame}


\end{document}